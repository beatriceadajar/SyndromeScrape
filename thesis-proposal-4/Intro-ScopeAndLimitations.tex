\section{Scope and Limitations}

This study focuses on syndromic surveillance, which uses syndromic data that is composed of identified signs and symptoms that are not necessarily confirmed through appropriate means of diagnoses by doctors. It will not focus on other means of epidemic intelligence such as indicator-based surveillance. 

This study will only focus on syndromic data that can be mined from health-related news articles published by local news sites. It will not tackle other sources of data such as infodemiological social media posts, electronic medical records, or other health-related media. 

Furthermore, this study will only focus on FASSSTER, a syndromic surveillance system developed by the Ateneo Java Wireless Competency Center. It will not seek to create a separate surveillance system altogether.
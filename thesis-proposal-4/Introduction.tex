\chapter{Introduction}

In compliance with the 2005 revisions of the International Health Regulation (IHR), countries were required to report disease outbreaks of international level to the World Health Organization (WHO) within 24 hours \cite{worldtechnical}. As such, many countries have employed disease surveillance techniques so that outbreak cases could be reported to the respective bodies as soon as possible \cite{hartley2010landscape}.

Many traditional disease surveillance systems have relied chiefly on data generated by various health care workers from private health institutions and the public health system \cite{rohart2016disease}. However, some systems from other countries have been shown to take weeks to properly confirm cases of disease before resolving to an appropriate response \cite{rohart2016disease}, showing a possible problem with timeliness.

Various public health systems have turned towards syndromic surveillance, where signs and symptoms are used to detect disease before confirming diagnoses \cite{mandl2004implementing}, making it a useful tool in the early detection and monitoring of disease outbreaks.

Syndromic surveillance systems use data from electronically available sources such as electronic medical records \cite{lazarus2001using}, but also utilize unstructured sources such as infodemiological (health-related) tweets \cite{espina2016towards}. Moreover, global surveillance systems such as GPHIN, EpiSPIDER, and HealthMap that utilize articles from news sites have also been shown to be effective in generating event-based outbreak information \cite{keller2009use}. Sources such as these provide current and localized information about outbreaks, even from areas that are relatively unnoticed by traditional public health efforts \cite{woodall1997official}.

As such, this study proposes to augment an existing syndromic surveillance effort in the Philippines called FASSSTER through web scraping local news sites for syndromic data. This would allow data readily created by news providers in the Philippines to be used in the early detection of diseases. 

\section{Research Questions}

This study seeks to determine how web scraping local news sites of syndromic data can augment FASSSTER, a syndromic surveillance initiative in the Philippines, in the early detection of diseases.

Specifically, it seeks to answer the following:
\begin{enumerate}
\item What methods of web scraping would allow for retrieving relevant syndromic data from local news websites and utilizing this information for syndromic surveillance?
\item Is the volume of news articles generated by local news sites sufficient enough to be used for syndromic surveillance?
\item How does web scraping syndromic data from local news sites augment FASSSTER in terms of relevance and reliability of data? 
\end{enumerate}
\section{Objectives of the Study}

This study aims to utilize web scraping techniques to retrieve relevant syndromic data from local news sites for use in syndromic surveillance.

More specifically, it aims to do the following:
\begin{enumerate}
\item Identify methods of web scraping local news sites to retrieve syndromic information that can be useful for the early detection of diseases.
\item Demonstrate that the volume of news articles generated by local news sites is sufficient for use in syndromic surveillance. 
\item Identify the value of web scraping local news sites in terms of retrieving relevant and reliable syndromic data.
\end{enumerate}
\section{Scope and Limitations}

This study focuses on syndromic surveillance, which uses syndromic data that is composed of identified signs and symptoms that are not necessarily confirmed through appropriate means of diagnoses by doctors. It will not focus on other means of epidemic intelligence such as indicator-based surveillance. 

This study will only focus on syndromic data that can be mined from health-related news articles published by local news sites. It will not tackle other sources of data such as infodemiological social media posts, electronic medical records, or other health-related media. 

Furthermore, this study will only focus on FASSSTER, a syndromic surveillance system developed by the Ateneo Java Wireless Competency Center. It will not seek to create a separate surveillance system altogether.
\section{Significance of the Study}

The study utilizes computer science concepts such as web scraping, support vector machines, latent Dirichlet allocation, relational databases, and system integration. Local news sites will be web scraped to retrieve news articles. These articles will then be subjected to support vector clustering to separate those that contain syndromic data from those that do not, then be further subjected to topic modeling through latent Dirichlet allocation. Information from this will then be stored in a relational database, and shall then be integrated into FASSSTER, an existing syndromic surveillance system.

This study could contribute to the improvement of syndromic surveillance systems employed in the Philippines. News sites generate relevant and crucial data that relate to health events, and this study aims to make use of this data for the earlier detection of diseases. This proposed system has the potential to give way to better syndromic surveillance efforts in the country. 